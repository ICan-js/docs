\chapter{Fundamentação Teórica}
\label{ch:fundamentacao}
\par Neste capítulo ser\~ao fundamentados os conhecimentos b\'asicos para o entendimento do trabalho.

\section{Deficiência}

De acordo com o censo do IBGE, realizado em 2010, cerca de 6.2\% da população brasileira possui algum tipo de deficiência. E a necessidade de inclusão destas pessoas na sociedade é extremamente importante. Do grupo citado anteriormente, cerca de 1.3\% tem algum tipo de deficiência auditiva, e 1.1\% tem deficiências auditivas

Para aqueles com deficiência auditiva, a comunicação e inclusão pode ser feita através da Linguagem Brasileira de Sinais, segunda língua oficial do Brasil desde 2005. Mas, pode-se encontrar problemas com a comunicação através de LIBRAS principalmente pelo fato de, boa parte dos ouvintes não falar esta língua o que acarreta também na baixa utilização desta em diversos meios de comunicação. Um ponto importante apontado no documentário feito pela TVE RS, é que, pessoas com deficiência auditiva, normalmente são alfabetizadas somente com LIBRAS, por terem muita difículdade e falta de estrutura para o aprendizado da Língua Portuguesa. Ainda de acordo com o documentário, para as pessoas com deficiências motoras há os recursos de tecnologias assitivas, que aumentar a facilidade do acesso destas pessoas aos meios sociais, principalmente os digitais.

% Precisa ficar realizando definições de cada uma das deficiências que serão tratadas no trabalho ?? Pensando hoje, acho que deveria sim! 
%% Caso precise, qual o nível de profundidade necessário ? ? ? Não tão profundo, mas a ponto de demonstrar a qual público o trabalho é destinado

\section{Tecnologias assistivas}

% NTAAI -> Verificar como citar esta fonte ! ! !
% Verificar se devo citar tecnologias assistivas, ou no meu caso, recursos assistivos ! ! !
Uma das formas de realizar a inclusão social de pessoas com deficiência é através da inclusão digital, utilizando técnologias assitivas, estes que visam ampliar as habilidades presentes no indivíduo, não o forçando a ter características específicas para a inclusão (NTAAI, 2016). As tecnologias assitivas, de acordo com o Núcleo de Tecnologia Assistiva, Acessibilidade e Inovação da Universidade de Brasilia, podem ser divididas em dois grupos, os recursos, que representam equipamentos que expandem as habilidades dos indivíduos com deficiência, e os serviços, que normalmente são aqueles relacionados a facilitação e capacitação para o uso correto dos recursos assistivos.

% Adicionar dados aqui que fazem jus ao que estou dizendo ! ! !
% Paragrafo abaixo deve antes ser reescrito para o novo escodo do trabalho
% Para aqueles que possuem deficiência motora, a inclusão digital vem através de \textit{mouses} adaptados, formas diferentes de utilizar teclados, ou até mesmo a utilização do computador por comandos de voz. E para os surdos ferramentas que ajudam no processo de interação já levando em consideração problemas com a alfabetização.


%%%%%%%%%%%%%%%%%%% Temporário (Coloquei aqui pois utilizo no trabalho, mas vou falar com o Giuliano)
% \section{Regressões}
% \par Um pequeno descritivo sobre regressões
%%%%%%%%%%%%%%%%%%%

\section{Análise de Regressão} % Ou Regressão linear

% MACHADO -> https://www.ime.usp.br/~fmachado/MAE229/AULA10.pdf
% PETERNELLI -> http://www.dpi.ufv.br/~peternelli/inf162.www.16032004/materiais/CAPITULO9.pdf
A análise de regressão estuda a relação entre uma variável dependente e outras independentes (MACHADO, 2015). Esta relação é representada através de um modelo matemático (MACHADO, 2015), este que pode ter diferentes formas sendo linear, quadrático, exponencial entre outros (PETERNELLI, 2003). 

% ToDo: Colocar um exemplo ? Se for colocar, qual será ? (02/02/2019)

%%%%%%%%%%%%%
% Antiga definição utilizada: Análise de regressão consiste na aplição de uma análise estatística com o objetivo de verificar a relação entre duas variáveis (PETERNELLI, 2003).
% Aqui eu posso colocar assim: Neste trabalho, após a verificação dos dados fez-se a utilização de regressões lineares
% Claro que isto deverá ser reescrito, até porque coloquei com as minhas palavras
% \par Os métodos de regressão são aqueles que buscam através de variáveis continuas ou categóricas estimar uma outra variável, podendo esta também ser continua ou categórica. Para este trabalho utilizou-se do método de regressão linear, este que básicamente busca através de uma função f(x) mapear a relação de duas variáveis linearmente separáveis.
%%%%%%%%%%%%%

% \section{Correlação de imagens digitais}

% http://www.dpi.inpe.br/~carlos/Academicos/Cursos/Pdi/pdi_estatisticas.html
% Se tudo funcionar da forma que estou pensando, vou colocar este conceito aqui (02/02/2019)
% Pois bem, não funcionou como eu esperada, assim vou manter este conceito de fora do documento (03/02/2019)

% \par A correlação é A, porém para este trabalho ela foi empregada em Imagens digitais

% \subsection{Regressão logistica}

% \par Uma função que divide o espaço, porém levando em consideração variáveis dependentes categóricas
%%%%%%%%%%%%%%%%%%%

\section{Inteligência artificial}

%%%%%%%%%%%%%%%%%%%%% Verificar a necessidade de escrever sobre inteligência artificial ! ! ! No momento (01/02/2019) eu acho que devo escrever sobre

% Rever todas as referências - Provavelmente reescrever o texto nas férias ! ! ! (Já arrumei melhor as coisas neste capítulo (01/02/2019)
% Von Zuben: ftp://ftp.dca.fee.unicamp.br/pub/docs/vonzuben/ea072_2s13/introducao_EA072_2s2013.pdf
% (Winston, 1992) Livro do Winston

Sistemas inteligentes de forma geral são aqueles que apresentam a capacidade de planejar e resolver problemas através de dedução e indução utilizando conhecimentos de situações anteriores (ZUBEN, 2013), e a inteligência artificial, é um campo da ciência e engenharia de computação (ZUBEN, 2013), que possibilitam a sistemas computacionais, perceber, raciocionar e agir (WINSTON, 1992).

% Augusto -> http://dcm.ffclrp.usp.br/~augusto/teaching/ami/AM-I-Conceitos-Definicoes.pdf
As técnicas computacionais mais utilizadas para o desenvolvimento e aplicação de inteligência artificial, são aquelas relacionadas ao aprendizado de máquina. Esta que é uma área que tem como objetivo principal, desenvolver técnicas que permitam aos sistemas adquirir conhecimento de forma automática e com estes conhecimentos tomar decisões (AUGUSTO, 2007).

% NG -> Curso do andrew
Para a realização do aprendizado de máquina, existem diversas técnicas, que vão de simples regressões estatísticas, até modelos complexos, como às redes neurais artificiais (RNA) (NG, 2016).
%%%%%%%%%%%%%%%%%%%%%

\section{Redes neurais artificiais}

% Haikin (2001) -> Livro (Redes Neurais: Princípios e Prática)
% Cintra -> Minicurso do INPE (2015)

% É um começo, mas ainda não estou feliz com o resultado.... (01/02/2019)
Redes neurais artificiais são sistemas computacionais que busca modelar o sistema cérebral natural humano, estas que são uma das formas de soluções de problemas apresentados dentro do âmbito de inteligência computacional (CINTRA, 2015).

%% Não creio que isto esteja escrito da melhor forma, mas está melhor que o texto inicial (01/02/2019)
Por buscar modelar o cérebro humano, as RNAs utilizam como unidade básica de processamento, os neurônios artificiais (HAYKIN, 2001), da mesma forma que o cérebro utiliza os neurônios biológicos. % Preciso criar um complemento para ir para o próximo capítulo ? (01/02/2019)

%Acho que esta segunda parte não está combinando com o resto do texto, mas por agora vou manter aqui
% Por buscar modelar o cérebro humano, a unidade básica de processamento das RNAs são os neurônios (HAYKIN, 2001), estruturas estas que tem fortes ligações com o sistema biológico.

%%%%%%% Partes antigos
% Redes neurais artificiais (RNA) são modelos criados para representar a maneira como o cérebro realiza suas tarefas, sendo estas maciçamente paralelas e distribuidas (HAYKIN, 2001). Ainda de acordo com Haykin (2001), estes modelos para ter bons desempenhos são representados normalmente através de interligações maciças de células computacionais simples, denominadas de neurônios. 
% Veja que, para a criação destes modelos, há uma grande inspiração em aspectos e funcionamentos do cérebro humano (CINTRA, 2015)
% Acho que isto aqui deve ser complementado ! ! ! Com toda a certeza!
% Cintra - Minicurso INPE (2015)
% Como descrito na seção anterior, uma das áreas de aplicação do aprendizado de máquina mais avançadas atualmente são as RNA, que imitam principalmente aspectos do funcionamento do corpo humano, neste caso, o cérebro e suas redes neuronais (CINTRA, 2015). 
%%%%%%%

\subsection{Neurônio biológico}

% Falar sobre sinapses ! ! ! -> Já falei sobre isto ! ! !
% Falar sobre como aprendemos ! ! ! -> Será  mesmo necessário ? ? ?

% Nunes -> Livro
Todo o processamento de informações no cérebro humano, é feito através de elementos biológicos de processamento, que operam em paralelo para a produção de ações apropriadas para cada estímulo recebido pelo corpo. A célula base do sistema nervoso cerebral é o neurônio (Figura 1), e sua principal função é conduzir impulsos (Representando os estímulos) levando em consideração as condições do corpo e assim produzindo ações. Os neurônios também são os resposáveis pelos atos do pensamento e armazenamento de informações (NUNES et al, 2016).

Os neurônios podem ser divididos em três partes elementares, os dendritos, que captam de forma continua os impulsos vindos de outros neurônios, o corpo celular, que processa todas as informações captadas e os axônios que enviam as informações processadas no corpo celular para outros neurônios.

% https://www.researchgate.net/figure/Figura-24-Ilustrativo-de-um-neuronio-biologico_fig17_303369695
\image{1.4}{neuronio-biologico.png}{Ilustração do neurônio biológico}{bioneuron}

% SHEPERD -> Artigo
% XAVIER -> Artigo (Divulgado no Alô Ciência) !
% Consultar o livro do Hebb apenas para ter uma certeza!
Estima-se que a rede neural cerebral, possui cerca de 100 bilhões de neurônios, cada um destes mantendo conexão com uma média de 6.000 outros neurônios, gerando cerca de 600 trilhões de conexões (SHEPHERD, 1990). A região de conexão entre os neurônios são chamadas de sinapses, estas que como apresentado por Donald Hebb em 1949, em seu livro \textit{The Organization of Behavior} são fortalecidas todas as vezes em que são utilizadas.

% A regiã entre os neurônios são chamados de sinapses, estas que como apresentado por Donald Hebb em 1949, em seu livro \textit{The Organization of Behavior} tem os caminhos fortalecidos toda vez que é utilizado
% , assim, pode-se entender que, neurônios tem propensões para certas atividades, quando os neurônios utilizados por esta tem suas sinapses bem fortalecidas (XAVIER, 2017).

A Figura 2 demonstra um exemplo de uma pequena parte das redes neuronais responsáveis pelo córtex auditivo.

% https://pt.wikipedia.org/wiki/Ficheiro:Cajal_actx_inter.jpg
\image{0.35}{nn_cortex.jpg}{Rede neural do córtex auditivo}{nncotex}

%%% Validar as informações abaixo
% Citar assim Hodgkin e Huxley (1952)  ou (Hodgkin; Huxley, 1952) ? ? ?

% McCulloch -> Artigo
% (HODGKIn, HUXLEY, 1952) ? ? ? 
A representação inicial deste conjunto de neurônios em sistemas de computação foram implementadas através de circuitos eletrônicos, com apresentado por McCulloc e Pitts (1943), estes que foram utilizados como base para a criação dos modelos de neurônios artificiais apresentados por Hodgkin e Huxley (1952).
%%%

\subsection{Neurônio artificial}

Como citado anteriormente, os neurônios artificiais, são modelos computacionais para a representação do neurônio biológico nas RNAs, e da mesma que em um neurônio biológico, a representação deste é feita com três elementos básicos (HAYKIN, 2001):  

\begin{itemize}
	% Posso colocar Conjunto de sinapses, representado por Xn... ? (02/02/2019)
	\item Conjunto de sinapses, cada uma caracterizada por um peso, este que indica a relevância de cada valor de entrada;
	\item Somador, ou combinador linear, que faz a ponderação dos valores de entrada com as respectivas sinapses do neurônio;
	\item Função de ativação utilizada para restringir os valores de saída do neurônio.
\end{itemize}

Ainda de acordo com Haykin (2001), a estes modelos neuronais pode-se aplicar um \textit{bias}, este que será o responsável pelo aumento ou diminuição dos valores de entrada da função de ativação. Em termos matemáticos, pode-se descrever um neurônio k (Figura 3) com as seguintes equações (HAYKIN, 2001):

\begin{equation}
	u_{k} = \sum_{j=1}^{n} w_{kj} x_{j}
\end{equation}
e
\begin{equation}
	y_{k} = f(u_{k} + b_{k})	
\end{equation}
onde $ x_{1}, x_{2}, ..., x_{n} $ são os sinais de entrada; $ w_{k1}, w_{k2}, ..., w_{kn} $ são os pesos sinápticos do neurônio k; $ u_{k} $ é a saída do combinador linear; $ b_{k} $ é o \textit{bias}; $ f(u_{k} + b_{k}) $ a função de ativação; e $ y_{k} $ representa a saída do neurônio

% Os neurônios artificias, que são modelos de representações dos neurônios biológicos compoem a RNA. O principal modelo de neurônio artificial utilizado, mesmo em arquiteturas mais atuais, é o proposto por McCulloch e Pitts em 1943 (Figura 3). Neste há componentes que fazem referência direta ao neurônio biológico visto anteriormente.

% Imagem adaptada de Haykin (2001)
\image{0.1}{nn_mculloch.jpg}{Modelo de Neurônio artificial}{nn_mcculloch}

A partir da Figura 3 é possível realizar uma comparação entre cada um dos elementos do neurônio artificial e biológico. Os sinais de entrada, advindos do meio externo, normalmente uma aplicação, são análogos aos impulsos elétricos captados pelos dendritos no neurônio biológico.  Os pesos sinápticos representam a importância do sinal recebido para o neurônio, o que representa as ponderações exercidas pelas junções sinápticas do modelo biológico, ou seja, a força do caminho entre as sinapses, citados anteriormente. O campo de somatório junto a função de ativação, representam o corpo celular do neurônio biológico, é nesta parte que os resultados criados pelo neurônio são calculados (NUNES et al, 2016).

\subsection{Arquiteturas de rede} 
% Eu estou deixando o texto da maneira como está, porém vou fazer mudanças depois (03/02/2019)
% Eu mudei muita coisa aqui, estava bastante incomodado com os resultados desta seção (03/02/2019)

A arquitetura de uma RNA define como seus vários neurônios estão dispostos em relação um ao outro \cite{DaSilva2016}. Para Haykin (2001), existem três classes de arquiteturas de redes neurais fundamentalmente diferentes.

% Colocar abreviaturas nestes nomes ?
\subsubsection{Redes alimentadas adiante com camada única}

As redes alimentadas adiante com camada única (Figura 4) são a forma mais simples de uma rede, onde são apresentadas duas camadas, a camada de entrada, que básicamente recuperam os dados do meio externo e a camada de saída, esta constituida de neurônios que processam os dados e retornam os resultados, não havendo caminho inverso. 

% \image{1.0}{r_a_ad_com_mult_camadas.png}{Redes alimentadas adiante com camada única}{nn_per}

% Comentar a imagem ? ?

\subsubsection{Redes alimentadas diretamente com múltiplas camadas}

% Falar aqui sobre a camada totalmente conectada é melhor! Acho mais interessante

As redes alimentadas diretamente com múltiplas camadas (Figura 5) tem em sua estrutura a presença de uma ou mais camadas ocultas, onde estão presentes conjuntos de neurônios, que são básicamente são responsáveis por permitir que a rede extraia características mais complexas dos dados captados na camada de entrada, assim permitindo que a rede possa aprender características globais (CHURCHLAND; SEJNOWSKI, 1992), não ficando limitada apenas aos dados utilizado em seu treinamento.

% \image{1.0}{r_a_de_com_mult_camadas.png}{Redes alimentadas diretamente com múltiplas camadas}{nn_mlp}

% Comentar a imagem ? ?

\subsubsection{Redes recorrentes}

Em Redes recorrentes (Figura 6) existem as camadas de entrada e saída, porém nestas os valores de saída de um neurônio pode ser utilizado como entrada para sí próprio ou mesmo algum outro neurônio da mesma camada.

% \image{1.0}{r_rec.png}{Redes recorrentes}{nn_rr}

% \begin{itemize}
% 	\item Redes alimentadas adiante com camada única: Possuem a camada de entrada conectada unicamente a camada de saída;
% 	\item Redes alimentadas diretamente com múltiplas camadas: Apresentam camadas ocultas, que são utilizadas para extração de características mais complexas do contexto onde está inserida;
% 	\item Redes recorrentes: Operam com laços de realimentação, onde a saída de um neurônio é também sua entrada.
% \end{itemize}

% (SILVA; SPATTI; FLAUZINO, 2010) -> Livro
% Um exemplo muito comum de redes alimentadas adiante com camada única são os \textit{Perceptrons} (Figura 4), estas que basicamente são compostas pela camada de entrada, que não realizam operações nos dados de entrada e a camada de saída, que neste caso são compostas por conjuntos de neurônios, que operam sobre as saídas da camada de entrada e realizam as saídas da rede (SILVA; SPATTI; FLAUZINO, 2010).

% Para as redes alimentadas diretamente com múltiplas camadas, o exemplo comum são as \textit{Multi-Layers Perceptrons} (Figura 5)

%As RNAs podem ser constituídas de alguns ou mesmo vários neurônios artificiais, que são organizados através de camadas, estas que por sua vez definem a arquitetura da rede neural artificial. Para Haykin (2001) a arquitetura da rede neural é especifica para cada aplicação, e pode ser composta por diferentes níveis de camadas ocultas, função de ativação e algoritmo de aprendizado, podendo variar também a quantidade de neurônios que está sendo empregado na solução do problema.

%Mesmo havendo diversas variações de arquiteturas para diferentes problemas, normalmente encontra-se em uma RNA três tipos de camadas, sendo uma a camada de entrada, esta responsável em apenas receber os dados, uma camada oculta, que recebe os dados da camada de entrada e processa em um conjunto de neurônios e a camada de saída de resultados (PONTI, 2017). 

%Um exemplo de RNA que apresenta cada uma destas camadas citadas são as \textit{Multi-Layer Perceptron} (Figura 4), que além de possuirem uma camada de entrada e uma de saída, podem apresentar diversas camadas ocultas.

%\image{1.0}{mlp.png}{Rede de múltiplas camadas}{nn_mlp}

%Mas veja que, há arquiteturas que não seguem este formato, como é o caso das redes \textit{Perceptron} (Figura 5), que básicamente possuem apenas a camada de entrada e saída, e neste caso, todo o processamento ocorre na camada de saída.

%\image{1.0}{perceptron.png}{Rede de alimentação direta e camada simples}{nn_perceptron}

% Estudar mais sobre arquitetura de rede ! ! !
%%%%%%%%%%%%%%%%%%%%%%%%%%%%%%
% \image{0.83}{nn_generic.png}{Rede neural artificial simples}{nn_generic}

% Mas há redes neurais artificiais como os \textit{Perceptrons} (Figura 5) que possuem apenas duas camadas, a camada de entrada, e a camada de saída, sendo que, o processamento dos dados neste tipo de rede é feito na camada de saída.

% \image{1.0}{perceptron.png}{Rede de alimentação direta e camada simples}{nn_perceptron}

% Buscar referências para a MLP
% E também poder haver redes com diversas camadas, como é os caso das \textit{Multi-Layer Perceptron} (Figura 6) que possuem as camadas de entrada e saída, e um conjunto de camadas ocultas que podem variar de uma única camada ou até mesmo dezenas delas, dependendo do problema a ser resolvido.

% \image{1.0}{mlp.png}{Rede de múltiplas camadas}{nn_mlp}
%%%%%%%%%%%%%%%%%%%%%%%%%%%%%%

% Em uma RNA simples, tem-se normalmente três camadas apenas (Figura 4), sendo uma a camada de entrada, esta responsável em apenas receber os dados, uma camada oculta, que recebe os dados da camada de entrada e processa em um conjunto de neurônios e a camada de saída de resultados (PONTI, 2017).

% Criar uma imagem que identifica cada uma das camadas que citei
% Acho que assim fica mais fácil entender o que está acontecendo


% Porém veja que, em arquiteturas simples como as redes \textit{Perceptron} (Figura 5), há apenas a camada de entrada e a camada de saída, não havendo camadas ocultas

% Este tópico vai ficar aqui, mas pode ser algo temporário, estou pensando sobre (02/02/2019)
% O que pensei foi, no momento que falar das camadas já falar da arquitetura de redes neurais, acho bacana fazer isto aqui
% As RNAs podem ser contituídas de alguns ou mesmo vários neurônios artificiais, que são organizados através de camadas, estes que por sua vez definem a arquitetura da rede neural. Para Haykin (2001), a arquitetura da rede 


% ToDo: Tentar buscar referência para confirmar esta afirmação do primeiro linha
% As RNAs podem ser contituídas de alguns ou mesmo vários neurônios artificiais, que são separados através de camadas, cada uma delas com uma função específica. Em uma RNA simples, tem-se normalmente três camadas apenas (Figura 4), sendo uma a camada de entrada, esta responsável em apenas receber os dados, uma camada oculta, que recebe os dados da camada de entrada e processa em um conjunto de neurônios e a camada de saída de resultados (PONTI, 2017).


% ToDo: Veja que isto deverá ser revisado
% Veja ainda que, a quantidade de camadas pode representar o poder de solução de problemas de uma rede, de acordo com PONTI (2017), RNAs de estruturas muito simples, como os \textit{Perceptrons}, que possuem apenas um neurônio, não são capazes de resolver problemas com complexidades não lineares.

% ToDo: Depois que comecei a ler sobre teoria do aprendizado estatístico, tudo o que eu escrevi não tem um total sentido, para trocar esta parte
% busque referências de aprendizado estatístico antes de continuar (02/02/2019)
% Através destes neurônios artificiais as RNAs são compostas com dezenas ou até mesmo centenas de neurônios, dependendo exclusivamente da complexidade do problema a ser resolvido. As RNA, ainda são divididas em camadas, onde cada uma delas tem uma função específica, uma RNA simples, normalmente tem-se três camadas apenas (Figura 4), uma camada de entrada, uma camada oculta, que contém um conjunto de neurônios e a camada de saída dos resultados.

% Na camada de entrada, os dados são recebidos e enviados para a camada oculta, na camada oculta, os dados são processados pelos neurônios, e seus resultados são unidos na camada de saída (PONTI, 2017).

% Um tipo muito comum de rede neural com esta estrutura básica são os chamados \textbf{Perceptrons}, que é uma rede neural simples, criada por Frank Rosenblat em 1957. Porém redes neurais com apenas uma camada de processamento (Camada oculta) não são aplicáveis em diversos casos, principalmente pela limitação no nível de complexidade de problemas que podem ser resolvidos com este tipo de rede (PONTI, 2017). Problemas como classificação de imagens, reconhecimento de objetos e detecção de fraudes podem ser um grande desafio para estes tipos de arquitetura, desta forma, mais camadas devem ser inseridas para tratamentos mais sofisticados dos dados, gerando resultados mais acertivos e principalmente, resolvendo problemas mais complexos. 

% Falar sobre as diferenças no processo de aprendizado (01/02/2019)
% Aqui falo de forma geral ? Preciso ir a fundo na questão ?? (01/02/2019)
% Talvez aqui eu possa introduzir erro, otimização, feed forward....estou pensando sobre isto (02/02/2019)
\subsection{Processo de aprendizado}

Um dos pontos mais relevantes das RNAs é seu poder de generalização, assim, após aprender a realizar alguma atividade, levando em consideração um determinado conjunto de dados, estas redes conseguem realizar atividades com diferentes conjuntos de dados. Porém isto exige um processo de treinamento bem definido, seguindo um algoritimo, este algoritimo é o processo de treinamento. (NUNES et al, 2016).

Estes processos de treinamento podem adotar diferentes estratégias para ensinar as RNA, e cada estratégia gera um algoritimo de aprendizado diferente, sendo os principais, os algoritimos de aprendizado supervisionado e não-supervisionado.

\subsubsection{Aprendizado supervisionado}

% OSÓRIO -> fia99 (http://osorio.wait4.org/oldsite/IForumIA/fia99.pdf)
No aprendizado supervisionado, o usuário indica o comportamento que a RNA deve apresentar dado um conjunto de dados qualquer, desta forma, a RNA pode ir ajustando os pesos sinápticos de seus neurônios com o objetivo de produzir o mesmo resultado apresentado pelo usuário (OSÓRIO, 1999). Levando em consideração um problema de classificação, onde um conjunto de dados é apresentado para a RNA, e ela deve informar ao usuário o que cada um dos dados daquele conjunto representa.

\subsubsection{Aprendizado não-supervisionado}

% NG -> Neste caso é o curso do Andrew NG
O aprendizado não supervisionado é completamente o oposto do supervisionado, neste é apresentado para a RNA apenas o conjunto de dados, e a RNA se encarrega de aprender sobre aquele conjunto de dados. Este tipo de aprendizado pode ser utilizado para deixar a RNA identificar os padrões presentes nos dados, e tirar informações destes padrões (NG, 2013). 

\section{Deep Learning}

O \textit{Deep Learning} apresenta uma abordagem diferente para os problemas resolvidos com técnicas de RNA, porém, no caso de DL, muitas camadas são empregadas (GOODFELLOW, 2016) nas arquiteturas das redes neurais (Figura 5).

\image{0.6}{snn_vs_dl.png}{Diferenças entre redes neurais simples e Deep Learning}{ml_vs_dl}

A utilização de múltiplas camadas, cada um com dezenas de neurônios, permitiu as técnicas de DL chegarem ao estado-da-arte em muitos problemas que envolvem o AM (SHANKAR, 2017).  Além disto, de acordo com Andrew NG, o processo de aprendizado destas redes melhora muito com o aumento dos dados (Figura 6), diferente do que ocorria com arquiteturas e algoritimos de aprendizado de máquinas antigos.

\image{0.4}{dl_vs_dados.png}{Deep Learning X Quantidade de dados}{dl_vs_dados}

Ainda de acordo com Andrew, isto ocorre pois ao utilizar múltiplas camadas, diversos recursos são captados dos dados, fazendo com que o processo de aprendizado se torne eficaz, e tende a melhorar ainda mais com o aumento da quantidade de dados utilizados no processo de treinamento. 

A utilização de múltiplas camadas, permitiram que diferentes técnicas pudessem ser utilizadas dentro de uma rede neural, e isto fez com que diversas arquiteturas, para os mais váriados fins fossem criados.

\subsection{Redes neurais convolucionais}

% SAVARESE, 2018 -> https://web.stanford.edu/class/cs231a/lectures/intro_cnn.pdf
% ARAÚJO, 2017 -> Redes neurais convolucionais com tensorflow: Teoria e Prática 
Redes neurais convolucionais, do inglês, \textit{Convolutional Neural Network} são um tipo de rede neural profunda, especializadas em análise de elementos visuais, tais como imagens e vídeos (SAVARESE, 2018).  Sua especialidade em dados visuais permitiu um grande avanço nas áreas de visão computacional, especialmente por estar serem mais fáceis de treinar, quando comparado a redes neurais comuns em trabalhos com imagens (ARAÚJO, 2017).

Um dos primeiros modelos de CNN propostos foi a LeNet (LECUN et al, 1998), proposta por Yann LeCunn em 1998, e mesmo com a evolução dos conceitos deste tipo de rede, os conceitos apresentados por LeCun continuam sendo aplicados. Nesta arquitetura, uma sequência de camadas convolucionais, de \textit{pooling} e totalmente conectadas são utilizadas (ARAÚJO, 2017).

As camadas convolucionais, que são a grande diferença das CNN para outros tipos de RNA, trabalham como filtros, recuperando apenas pontos importantes da imagem para a classificação, isto através de uma matriz de pesos que é utilizada nas convoluções (ARAÚJO, 2017). Após o filtro realizado por esta camada, as imagens resultantes do filtro são passadas para a camada de \textit{pooling}, estas camadas que básicamente reduzem a dimensionalidade das resultantes. Por fim, as camadas totalmente conectadas são as responsáveis em realizar a multiplicação ponto a ponto dos sinais recebidos (imagens) e aplicar uma função de ativação, que produzirá a probabilidade de cada uma das classes esperadas na classificação (ARAÚJO, 2017).

A Figura 7 demonstra a arquitetura de LeCun sendo utilizada para a classificação de imagens de tumores, podendo ter como resultado às classes \textbf{normal} ou \textbf{anormal}.

\image{0.6}{cnn_lenet.JPG}{Estrutura básica de CNN proposta por LeCun, 1998}{lecun_basic}

Veja que, o diferencial citado acima, na utilização das convoluções está justamente na quantidade de elementos que são utilizados para a classificação, em RNA comuns, ao realizar a classificação de imagens, deve-se ter de neurônios na RNA a mesma quantidade de píxels presentes na imagem a ser classificada, o que nas CNN não ocorre, exatamente por conta dos filtros que são criados (PONTI, 2017). 

\subsubsection{Camada de convolução}

\subsubsection{Camada de pooling}

% \subsubsection{Camada totalmente conectada} % Esta seção fica aqui pois faz parte das redes neurais convolucionais (02/02/2019), (03/02/2019) R: não! Isto faz parte de rede neural mesmo! 

\subsection{Mobilenet}

Falar sobre a arquitetura Mobilenet, assim como sua forma de convolução

\subsection{Posenet}

\par Posenet é um tipo de CNN, para a identificação em tempo real de pontos do corpo dos usuários, o modelo desenvolvido para esta rede é extremamente robusto e permite a identificação das poses mesmo quando há problemas com luz e iluminação do ambiente que está sendo levado em consideração na classificação (KENDALL, 2015). 

\subsection{Transferência de aprendizado}

% Em aprendizado profundo existe a necessidade de datasets com centenas de milhares de imagens...
\par Explicar sobre transferência de aprendizado

\section{Tecnologias}

Esta seção demonstra as tecnologias utilizadas durante a implementação do presente trabalho.

\subsection{Tensorflow.js}

% https://www.tensorflow.org/about/bib
% Agora tem referências próprias ! ! !
Tensorflow.js é uma biblioteca para a linguagem de programação Javascript, que permite a implementação de modelos de AM com grande facilidade de expressão. A biblioteca é flexível e pode ser usada para expressar uma ampla variedade de algoritmos, incluindo algoritmos de treinamento e inferência para modelos de DL, e tem sido usado para conduzir pesquisas e implantar sistemas de aprendizado de máquina nas mais diversas áreas, envolvendo trabalhos como reconhecimento de fala, visão computacional e robótica (ABADI et al, 2015). 

\subsection{Google Colaboratory}

% Eu que fiz esta definição ! ! !
Colaboratory é uma ferramenta criada pelo Google, que permite a fácil execução de algoritimos de aprendizado de máquina. O ambiente é criado sobre o pacote Jupyter, um ambiente interativo e simples para a execução de código, com a diferença de que, no Colaboratory, toda a execução pode ser feita utilizando máquinas disponibilizadas pelo Google.
