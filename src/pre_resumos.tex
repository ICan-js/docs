% resumo em português
\setlength{\absparsep}{18pt} % ajusta o espaçamento dos parágrafos do resumo
% --- resumo em português ---
\begin{resumo}
    \par Pessoas com qualquer tipo de deficiência tem o direito de uma vida normal, sem nenhum tipo de restrição por conta de suas condições Físico-Mentais. Esse direito garante as pessoas com deficiência uma vida digna. Recursos assistivos representam uma garantia de dignidade e acessibilidade à vida de pessoas com deficiência. Cada um desses recursos assistivos são criados para potencializar as características já presentes nos indivíduos, diminuindo as barreiras existentes por características que não possuem. No entanto, os recursos assistivos devem ser adaptados para cada contexto de aplicação, o que pode ser um problema, quando cada contexto de aplicação possui um preço de aquisição diferente. Com os recursos tecnológicos atuais é possível criar recursos assistivos adaptáveis a variados contextos, desta forma, o objetivo deste Trabalho é desenvolver recursos assistivos para \textit{web} utilizando técnicas de Aprendizado Profundo. Para isso os modelos de redes neurais convolucionais MobileNet e PoseNet foram utilizados no desenvolvimento dos recursos assistivos e sua aplicação em páginas \textit{web} apresentaram bons resultados na generalização, facilidade de uso e aplicabilidade.
    \vspace{\onelineskip}
    \noindent
    \\
    \textbf{Palavras-chave}: Recursos assistivos, Aprendizado Profundo, Redes Neurais, Redes Neurais Convolucionais, JavaScript
\end{resumo}

\begin{resumo}[Abstract]
    \begin{otherlanguage*}{english}
        \par People with any kind of disability have the right to a normal life, without any kind of restriction because of their Physical-Mental conditions. This right guarantees people with disabilities a decent life. Assistive resources represent a guarantee of dignity and accessibility to the lives of people with disabilities. Each of these assistive features is created to enhance the characteristics already present in individuals, reducing the barriers by characteristics they do not have. However, assistive features must be tailored to each application context, which can be a problem, when each application context has a different acquisition price. With the current technological resources, it is possible to create assistive resources adaptable to various contexts, in this way, the objective of this work is to develop assistive web resources using Deep Learning techniques. For this, the convolutional neural network models MobileNet and PoseNet were used in the development of the assistive resources and their application in web pages presented good results in generalization, ease of use and applicability.
	    \vspace{\onelineskip}
	    \noindent
	    \\
	    \textbf{Keywords}: Assistive Resources, Deep Learning, Neural Network, Convolutional Neural Network, JavaScript % Keywords, abstract, english.
    \end{otherlanguage*}
\end{resumo}
