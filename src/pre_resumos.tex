% resumo em português
\setlength{\absparsep}{18pt} % ajusta o espaçamento dos parágrafos do resumo
% --- resumo em português ---
\begin{resumo}
    % Apresentação concisa dos pontos relevantes do documento deve ser exposta no resumo. No presente caso o resumo será informativo, assim deverá ressaltar o objetivo, a metodologia, os resultados e as conclusões do documento. A ordem desses itens depende do tratamento que cada item recebe no documento original. O resumo deve ser composto por uma seqüência de frases concisas, afirmativas e não em enumeração de tópicos. Deve ser escrita em parágrafo único e espacejamento de 1,5. A primeira frase deve ser significativa, explicando o tema principal do documento. Deve-se usar o verbo na voz ativa e na terceira pessoa do singular. Quanto a sua extensão, o resumo deve possuir de 150 a 500 palavras.   
    \vspace{\onelineskip}
    \noindent
    \textbf{Palavras-chave}: % palavras chaves, resumo, português.
\end{resumo}

% resumo em inglês
\begin{resumo}[Abstract]
    \begin{otherlanguage*}{english}
        %O abstract é o resumo da obra em língua estrangeira, que basicamente segue o mesmo conceito e as mesmas regras que o texto em português. Recomenda-se que para o texto do abstract o autor traduza a versão do resumo em português e faça, se necessário, os ajustes referentes à conversão dos idiomas. É importante observar que o título e texto NÃO DEVEM estar em itálico.
	    \vspace{\onelineskip}
	    \noindent
	    \\
	    \textbf{Keywords}: % Keywords, abstract, english.
    \end{otherlanguage*}
\end{resumo}