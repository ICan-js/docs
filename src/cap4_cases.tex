\newpage
\chapter{Resultados}

\par Neste capítulo serão apresentados os resultados do modelo gerado e as páginas \textit{web} criadas para testar os módulos implementados no ICan.js

\section{Tradução de Libras para texto}

\par Nesta seção, os resultados obtidos com a transferência de aprendizado realizada no modelo MobileNet e também as páginas de testes deste RA são apresentadas.

\subsection{Resultados da transferência de aprendizado do modelo MobileNet}

\image{0.23}{resultados/resultados_rede_neural.jpg}{Resultado do retreino do MobileNet}{figure:historico_retreino}{Produção do Autor}

\par A Figura \ref{figure:historico_retreino} mostra a acurácia e a perda do modelo durante seu retreino, nota-se que, o processo de retreino levou 5 épocas para atingir bons resultados, isto, tanto na diminuição da perda, quanto no aumento da acurácia do modelo, que nos dois casos, nas primeiras épocas não estavam apresentando bons resultados.

\par O comportamento dos resultados na Figura \ref{figure:historico_retreino}, onde, mesmo os dados de treino sendo 100\% identificados a partir da segunda época e tendo poucas perdas, para os dados de teste, tem-se uma variação inversamente proporcional entre a crescente da acurácia e a diminuição da perda. 

\par Para avaliar o desempenho do modelo, além dos resultados da Figura \ref{figure:historico_retreino}, fez-se a utilização da matriz de confusão, esta que vincula as classes esperadas com a preditas em cada um dos valores presentes nos dados. Como citado anteriormente, os dados foram separados em dois conjuntos, o primeiro para treino/teste e o segundo para validação, isto para garantir que, o modelo gerado de nenhuma forma possui bons resultados somente por conta de um viés aos dados. Desta forma, duas matrizes de confusão foram criadas, uma para o conjunto de treino e teste (Figura \ref{figure:matriz_confusao_teste}) e outra para o conjunto de validação (Figura \ref{figure:matriz_confusao_validacao}).

\par A matriz de confusão da Figura \ref{figure:matriz_confusao_teste}, utilizou apenas os dados de teste, e mesmo nestes havendo imagens de participantes que foram apresentados ao modelo nos dados de treino, uma pequena quantidade de erro foi registrada nas classes Amigo e Desculpa.

\image{0.60}{resultados/matriz_de_confusao_teste.png}{Matriz de confusão do modelo MobileNet retreinado com dados de teste}{figure:matriz_confusao_teste}{Produção do Autor}

\par Já na matriz de confusão de validação (Figura \ref{figure:matriz_confusao_validacao}), que possui dados nunca antes apresentados para o modelo retreinado apresentou excelentes resultados, não errando em nenhuma das classes.

\image{0.60}{resultados/matriz_de_confusao_validacao.png}{Matriz de confusão do modelo MobileNet retreinado com dados de validação}{figure:matriz_confusao_validacao}{Produção do Autor}

\par É interessante notar que, o erro pode ter ocorrido para a matriz dos dados de teste (Figura \ref{figure:matriz_confusao_teste}) por conta de dados que, mesmo sendo filtrados, apresentam características muito diferentes das levadas em consideração pelo modelo, o que acaba não ocorrendo nos dados do conjunto de validação (Figura \ref{figure:matriz_confusao_validacao}).

\subsection{Aplicação do recurso assistivo desenvolvido}

\par Além da validação realizada nos resultados do MobileNet utilizado neste RA, uma aplicação\footnote{Disponível em: \url{https://icanjs-examples.netlify.com/escrita-de-texto/}} foi desenvolvida para validar a eficácia da biblioteca na distribuição de um RA que permite a escrita de textos em campos de páginas \textit{web}. Nesta aplicação espera-se que a biblioteca permita a escrita através de Libras em um campo de uma página \textit{web}.

\subsubsection{Página inicial}

\par Esta aplicação possui uma página, em seu início (Figura \ref{figure:case_libras_1}), há uma descrição dos objetivos da página e também sobre o projeto desenvolvido.

\image{0.70}{resultados/estudo-de-caso_escrita-de-texto/1.PNG}{Primeira parte da página da aplicação desenvolvido}{figure:case_libras_1}{Produção do Autor}   

\par Após a parte de apresentação, há uma seção que explica as formas de funcionamento (Figura \ref{figure:case_libras_2}) do exemplo, nessa são descritas as duas formas de funcionamento descritas durante o Capítulo \ref{ch:desenvolvimento}, a primeira (Exclusiva) transcrevendo o gesto capturado em cada \textit{frame} e a segunda (Contínua), que considera os \textit{frames} capturados em um intervalo de tempo para definir o gesto que deve ser transcrito.

\image{0.70}{resultados/estudo-de-caso_escrita-de-texto/2.PNG}{Seção de explicações sobre o funcionamento do exemplo da aplicação}{figure:case_libras_2}{Produção do Autor}   

\par Com o final das explicações o exemplo presente na aplicação é apresentado, disponibilizando as duas formas de funcionamento explicadas na Figura \ref{figure:case_libras_2}. A primeira forma de funcionamento não permite a geração de um texto contínuo, transcrevendo gesto a gesto para o campo da página (Figura \ref{figure:case_libras_3_1}). 

\image{0.65}{resultados/estudo-de-caso_escrita-de-texto/3.PNG}{Exemplo de funcionamento exclusivo}{figure:case_libras_3_1}{Produção do Autor} 

\par Já a segunda forma de funcionamento, presente na Figura \ref{figure:case_libras_3_2}, permite a escrita de textos contínuos, utilizando intervalos de tempo, o que também deixa seu uso agradável.

\image{0.65}{resultados/estudo-de-caso_escrita-de-texto/4.PNG}{Exemplo de funcionamento contínuo}{figure:case_libras_3_2}{Produção do Autor} 

\par Com isso, o objetivo da escrita de textos com Libras em campos de páginas \textit{web} foi atingido através do módulo implementado na biblioteca.

\section{Controle do cursor do mouse com movimentos da cabeça}

\par Esta seção apresenta a aplicação criada com os resultados obtidos na implementação do ICan.js e suas funcionalidades para o recurso assistivo de controle de \textit{cursor}.

\subsection{Aplicação do recurso assistivo desenvolvido}

\par Para realizar testes com o recurso assistivo implementado na biblioteca, fez-se a criação de uma aplicação\footnote{Disponível em: \url{https://icanjs-examples.netlify.com/controle-de-mouse/}}que consome os recursos da biblioteca. Nas subseções abaixo as páginas desenvolvidas serão apresentadas.

\subsubsection{Página inicial e calibração}

\par A página inicial da aplicação desenvolvida possui informações gerais do projeto e também da forma de funcionamento do mesmo, como exibido na Figura \ref{figure:case_cursor_tela_1}.

\image{0.47}{resultados/estudo-de-dado_controle-de-cursor/controle_do_mouse_1.PNG}{Tela inicial}{figure:case_cursor_tela_1}{Produção do Autor}

\par Na Figura \ref{figure:case_cursor_tela_1}, logo após a explicação do exemplo de execução da aplicação há um botão para que a execução seja iniciada, no momento em que o usuário clica no botão, a tela de calibração do modelo de regressão que será utilizado é exibida (Figura \ref{figure:case_cursor_tela_2}), nesta primeira tela de calibração, informações gerais são exibidas e então o usuário pode começar a calibração.

\image{0.40}{resultados/estudo-de-dado_controle-de-cursor/controle_do_mouse_2.PNG}{Tela inicial de calibração}{figure:case_cursor_tela_2}{Produção do Autor}

\par Com o usuário iniciando a aplicação, fica disponível para ele uma matriz 3X3 que ao ser clicada, salva a posição do \textit{cursor} e também do nariz do usuário. Aqui é esperado que, quando o usuário clicar nos pontos da matriz ele esteja apontando o nariz para a direção do ponto. Cada clique válido do usuário, faz uma barra de progresso no topo da tela ser atualizada (Figura \ref{figure:case_cursor_tela_2_2}).

\image{0.33}{resultados/estudo-de-dado_controle-de-cursor/controle_do_mouse_2_2.PNG}{Matriz de calibração e barra de progresso}{figure:case_cursor_tela_2_2}{Produção do Autor}

\par Ao término da coleta de pontos, o modelo é calibrado e salvo no \textit{sessionStorage}\footnote{Local para armazenamento temporário em navegadores} do navegador, ao mesmo tempo que, uma mensagem de finalização é exibida (Figura \ref{figure:case_cursor_tela_2_3}), nesta mensagem o usuário pode escolher continuar para a próxima tela ou mesmo recalibrar o modelo, sendo que, ao clicar nesta segunda opção, o processo visto anteriormente é recomeçado.

\image{0.40}{resultados/estudo-de-dado_controle-de-cursor/controle_do_mouse_2_3.PNG}{Aviso da finalização da calibração}{figure:case_cursor_tela_2_3}{Produção do Autor}

\subsubsection{Páginas de exemplos}

\par Com a finalização da calibração é exibido ao usuário uma página para seleção de exemplos (Figura \ref{figure:case_cursor_tela_3}), nesta está disponível uma página para testar a funcionalidade de \textit{scrolling} de páginas, isto feito através da leitura de um texto e outra para desenhar, que aproveita os recursos da camada \textit{Core} para construir uma aplicação de desenhos através de gestos.

\image{0.47}{resultados/estudo-de-dado_controle-de-cursor/controle_do_mouse_3.PNG}{Página de seleção de exemplos}{figure:case_cursor_tela_3}{Produção do Autor}

\par A página de leitura de texto (Figura \ref{figure:case_cursor_tela_4}) utiliza a função \textit{screenScroller}, sem nenhuma mudança, o resultado é uma \textit{div} que representa o \textit{cursor} criado que é movimentado através dos gestos.

\image{0.47}{resultados/estudo-de-dado_controle-de-cursor/controle_do_mouse_4.PNG}{Página de leitura de texto com a função \textit{screenScroller}}{figure:case_cursor_tela_4}{Produção do Autor}

\par Já na página de desenho, foi criado uma nova forma de aplicação das funcionalidades presentes do ICan.js, presentes na camada \textit{Core}, demonstrando que, além dos recursos já implementados na camada \textit{Common} é possível realizar a implementação de novas funcionalidades com a biblioteca.

\image{0.47}{resultados/estudo-de-dado_controle-de-cursor/controle_do_mouse_5.PNG}{Canvas de desenho criado no exemplo de desenho com o ICan.js}{figure:case_cursor_tela_5}{Produção do Autor}

\par Com isto, o objetivo da criação de um recurso assistivo para controle de \textit{cursor} com gestos do usuário foi atingido através das funcionalidades implementadas no ICan.js.

% \section{Experimentos}

% \par \textbf{O ORIENTADOR DISSE QUE, DA MESMA FORMA COMO FOI FEITO NA CAPÍTULO 3, A DIVISÃO PODE SER FEITA AQUI TAMBÉM! =D}

%% ?? ? ? ?
% Ainda estou com dúvida sobre como apresentar os resultados...

% Devo apresentar os resultados separado da mesma forma que o capítulo de desenvolvimento ? Ou posso fazer algo como demonstrar os resultados da rede, e então já a implementação da biblioteca, junto aos exemplos....
