% ----------------------------------------------------------
% Introdução (exemplo de capítulo sem numeração, mas presente no Sumário)
% ----------------------------------------------------------
\chapter[Introdução]{Introdução}
%\addcontentsline{toc}{chapter}{Introdução}
% ----------------------------------------------------------

% \par O trabalho é motivado pela necessidade / oportunidade de...
% \par Apresentar brevemente o problema para explicar a motiva\~c\~ao para o realizar.
% \par Recomendável a utilização de figuras e/ou tabelas.

Este capítulo demonstra a motivação para o desenvolvimento deste trabalho, os objetivos deste e a metodologia adotada.

\section{Motivação} % Problema
% Dica do Mineda: Colocar mais referências, gastar mesmo aqui.... Isto ajuda a fazer com que a importância do trabalho seja esclarecida.

% Esta motivação está dando a entender que o problema a ser resolvido é outro, NÂO, lembre-se que, objetivo é demonstrar que recursos assistivos criados com Deep Learning podem ser aplicados em tecnologias da WEB!!!! BOa

% Colocar também uma observação feita pelo Arakaki -> O que é caro ? O que é barato ?


% e a necessidade de importação \cite{ANDRIOLI2017}. Sendo que, o alto custo, é justificado principalmente pela necessidade da criação de equipamentos específicos para cada caso

% O alto custo, na maioria dos casos pode ser justificado pela necessidade de desenvolvimento e construção de equipamentos específicos cada cada caso, isto por 

% \par No Brasil, há uma grande dificuldade de acesso aos recursos de tecnologia assistiva, causadas por diversos fatores, a citar, o alto custo e a necessidade de importação \cite{ANDRIOLI2017}. O alto custo, na maioria dos casos pode ser justificado pela necessidade de desenvolvimento e construção de equipamentos específicos, o que acaba gerando um alto valor de compra.

% . A principal característica

% estas que deixam de lado as deficiências de lado e focam nas habilidades já presentes no indivíduo \cite{tve2016} para permit

% em suas habilidades \cite{tve2016} para permitir 

% \cite{tve2016}

% são as tecnologias assistivas, 

% \par Para serem independentes nem todos os deficientes precisam de algum auxílio, porém aqueles que necessitam utilizam 

% \par Uma das maneiras de permitir que deficientes sejam inclusos na sociedade e tenham uma vida autônoma é com a utilização de  recursos de tecnologias assistivas, estas aliadas a serviços de tecnologia assistiva. Isto porque, estes recursos assistivos deixam de lado a deficiência, e focam nas habilidades presentes no individuo.

% Neste âmbito surgem as tecnologias assistivas

% e todas estas pessoas necessitam de uma vida inclusiva e independente, não sendo impedidas de realizar suas atividades por conta de suas 

% E todas estas pessoas necessitam de uma vida independente e de inclusão (SARTORETTO; BERSCH, 2017).

% \par No Brasil, há uma grande dificuldade de acesso aos recursos de tecnologia assistiva, causadas por diversos fatores, a citar, o alto custo e a necessidade de importação \cite{ANDRIOLI2017}. O alto custo, na maioria dos casos pode ser justificado pela necessidade de desenvolvimento e construção de equipamentos específicos, o que acaba gerando um alto valor de compra.

% \textbf{SE EU FALAR DE RECURSOS ASSISTIVOS DE BAIXO CUSTO, O TRABALHO TERÁ DE BUSCAR INFORMAÇÕES SOBRE PREÇO....PENSE NISTO....}

% \textbf{AQUI SERÁ LEGAL FALAR SOBRE A GRANDE DIFICULDADE DE AQUISIÇÃO DE RECURSOS ASSISTIVOS, E MESMO SOBRE OS AVANÇOS QUE TEM SIDO FEITOS....}

% % Não sei no que isto implica
% \textbf{TALVEZ FALAR SOBRE A GRANDE QUANTIDADE DE TRABALHOS SOBRE RECURSOS ASSISTIVOS E MESMO ASSIM POUCA APLICAÇÃO... --> "TALVEZ" <--}

\par De acordo com o censo do IBGE, realizado em 2010, no Brasil, há cerca de 45 milhões de pessoas com algum tipo de deficiência e todas estas necessitam de uma vida independente e inclusiva, não sendo impedidas de realizar suas atividades por conta da deficiência.

\par Para uma parte deste público-alvo, a solução para a inclusão e independência são as tecnologias assistivas \cite{ANDRIOLI2017}, que através dos avanços tecnológicos criam possibilidades para as pessoas que possuem deficiência \cite{Bersch2017}. Este tipo de tecnologia deixa de lado as deficiências e foca apenas nas habilidades já presentes nos indivíduos \cite{tve2016}.

\par Porém no Brasil, mesmo com diferentes iniciativas surgindo ao longo dos anos \cite{Bersch2017} há uma grande dificuldade de acesso aos recursos de tecnologia assistiva \cite{ANDRIOLI2017}, causados por diversos fatores, a citar, o alto custo, por conta da grande necessidade da criação de equipamentos específicos para cada caso \cite{tve2016}, e também a necessidade de importação destes equipamentos \cite{ANDRIOLI2017}.

\par Por outro lado, tem-se técnicas de Aprendizado Profundo, que através da aplicação de redes neurais artificiais profundas, desde 2012 vem sendo apresentadas como o estado-da-arte para a solução dos mais diversos problemas, nos mais variados contextos \cite{Ponti2018, forbes2019}, isto ocorrendo principalmente por conta da alta capacidade de generalização e adaptabilidade destes algoritmos \cite{Camila2017, Ponti2018, forbes2019}.

\par Estas características fizeram com que a procura e aplicação destes algoritmos aumentasse exponencialmente, como pode ser visto na Figura \ref{figure:interesse_aprendizado_profundo}. Isto fez com que diversas empresas como Google e Facebook começassem a utilizar muito destas técnicas em seus sistemas e aplicações, o que impulsionou a área de uma forma nunca antes vista.

\image{0.8}{intro/interesse_deep_learning.jpeg}{Interesse em aprendizado profundo de 2012 à 2019}{figure:interesse_aprendizado_profundo}{\citeonline{google2019}}

\par Com empresas interessadas na rápida prototipação e implementação destas técnicas em seus serviços, diversas bibliotecas de códigos, nas mais variadas linguagens, que facilitam a aplicação dessas técnicas foram criadas, e muitas disponibilizadas como \textit{software} livre. Dentre as bibliotecas disponibilizadas, destaca-se \cite{pytorch2017, tensorflowjs2019, tensorflow2015-whitepaper, chollet2015}. Estas bibliotecas trouxeram benefícios não apenas para as empresas por trás delas, mas também para todas as comunidades com interesse em tais técnicas.

\par Para a área cientifica, veio a oportunidade de aplicação de tais técnicas para impulsionar diferentes estudos como em \cite{Frid-Adar2018, Caroline2016}, e também outras empresas se beneficiaram expandindo seus negócios e áreas de atuação.

\par Neste contexto, para o desenvolvimento de recursos assistivos mais precisos e acessivos, as técnicas de Aprendizado Profundo começaram a ser aplicadas, como é o caso de \citeonline{Magalh2018}, que desenvolve um reconhecedor de gestos de Libras estado-da-arte com tais técnicas. Mas, estas aplicações ainda estão muito restritas apenas ao desenvolvimento e geração de modelos de redes neurais profundas, fazendo com que seja necessário a implementação destes em sistemas assistivos, como a biblioteca de navegação apresentada por \cite{handsfree2019}. Assim, este trabalho é motivado pela possibilidade do desenvolvimento de recursos assistivos para páginas da \textit{web} aplicando técnicas de Aprendizado Profundo, trazendo mais precisão as ferramentas desenvolvidas e mais acessibilidade para as páginas da \textit{web}.

% \cite{handsfree2019}

% , destacando principalmente a alta personalização destes algoritmos 


% com multiplicas camadas, que desde 2012

% que desde 2012 tem crescido exponencialmente (Figura \ref{})



% que são atualmente o estado-da-arte da solução de problemas com aprendizado de máquina.

% \cite{forbes2019}.

% \par Por outro lado, tem-se técnicas de \textit{Deep Learning}, que são atualmente o estado-da-arte da solução de problemas com aprendizado de máquina (PONTI, 2017), \textbf{DIVERSOS MERCADOS DE APLICAÇÃO...} isto por conta da grande capacidade de generalização diante de diferentes conjuntos de dados. Um de seus grandes benefícios é a possibilidade de alta personalização frente a diferentes tipos de usuários e aplicações.

% \par Desta forma, este trabalho foi motivado pela possibilidade da realização de um estudo de casos, onde técnicas de \textit{Deep Learning} são aplicadas para possibilitar a criação de recursos de tecnologias assistivas de baixo custo.

% \textbf{OS TEXTOS ABAIXO DEVEM SER TRANSFORMADOS EM CONTEÚDO}


% \textbf{DESDE 2012, COM A COMPETIÇÃO IMAGENET VENCIDA POR UM ALGORITMO DE DEEP LEARNING, AS COISAS PARA ESTA ÁREA APENAS MUDOU, TENDO O SURGIMENTO DE DIVERSAS BIBLIOTECAS (TENSORFLOW, KERAS, PYTORCH, TFJS, caffe <- Todos estes podem ser citados!!!}



% \textbf{ASSIM, DIVERSAS APLICAÇÕES COMEÇARAM A SER REALIZADAS COM DEEP LEARNING...}

% https://glitch.com/edit/#!/handsfree-starter?path=README.md:1:0
% http://www.each.usp.br/petsi/jornal/?p=1496
% http://www.acessibilidadelegal.com/20-padroes.php

% Section Objetivo Geral!
\section{Objetivo Geral}

\par O objetivo geral deste trabalho é desenvolver uma biblioteca JavaScript que possa levar recursos assistivos desenvolvidos através de técnicas de Aprendizado Profundo para páginas da \textit{web}.

% \par Implementar recursos de tecnologias assistivas de baixo custo, para usuários com deficiências auditiva e motora, utilizando técnicas de \textit{Deep Learning}

% Section Objetivo Especifico!
\section{Objetivo Espec\'ifico}

\par Para a consecução do objetivo geral foram estabelecidos os seguintes objetivos específicos:

\begin{itemize}
    \item Desenvolvimento de um recurso assistivo que permite a movimentação do \textit{cursor} do \textit{mouse} em páginas \textit{web} através de gestos da cabeça;
    \item Desenvolvimento de um recurso assistivo que permite a escrita de texto em campos de páginas da \textit{web} através de gestos de Libras;
    \item Integração das ferramentas desenvolvidas em uma biblioteca JavaScript.
\end{itemize}

\section{Metodologia} % COMO !! 
% Reescrever a metodologia (24/03/2019)

\par A realização dos objetivos específicos deste trabalho são feitas através da aplicação de modelos de Aprendizado Profundo no desenvolvimento dos recursos assistivos para páginas da \textit{web}, todos estes consolidados através de uma biblioteca JavaScript.

\par Para o desenvolvimento do recurso assistivo de escrita de texto com gestos de Libras, dados serão coletados e um modelo MobileNet \cite{howard2017mobilenets} será treinado com os dados para possibilitar o reconhecimento de gestos e então traduzir estes para textos. Já no recurso assistivo para a movimentação do \textit{mouse}, através da aplicação do modelo PoseNet \cite{PoseNetMedium2019}, junto a regressões lineares, será feito o mapeamento dos gestos do usuário para posições do \textit{mouse} na tela.

\par A disponibilização e utilização destes modelos dentro da biblioteca desenvolvida neste trabalho, esta nomeada de ICan.js, será feita através de módulo Tensorflow para JavaScript \cite{tensorflowjs2019}.

% A realização dos objetivos específicos do trabalho é feita através da aplicação de modelos de DL no desenvolvimento das ferramentas. A linguagem de programação empregrada para o desenvolvimento das ferramentas é o Javascript, junto ao \textit{framework} de desenvolvimento de DL \textit{Tensorflow.js}

% Por contar com diferentes estudos de caso, todas as ferramentas são desenvolvidas de maneira modular, a permitir que, no momento da integração entre as ferramentas desenvolvidas, injeções de dependências sejam realizadas para tal.

% Cada uma das ferramentas desenvolvidas, faz a utilização de um modelo de DL, no caso do controle do \textit{mouse} o modelo Posenet (KENDALL, 2015) é utilizado, para facilitar a identificação de pontos faciais do usuário, e permitir que cada gesto seja mapeado em movimentos do \textit{mouse}, o reconhecimento de voz, por sua vez, é feito com o \textit{Web Speech API}, uma \textit{API} livre que facilita a sintetização de som em texto. Por fim, para a tradução de LIBRAS em texto, será utilizado uma rede neural convolucional (LECUN et al, 1998), que apresenta bons resultados na classificação de imagens, junto a um conjunto de imagens de LIBRAS criado pelo autor.

% Retirado apenas para ganhar tempo! <- Esta ainda é a oficial e deve ser levada em consideração ! ! !
% o retreino do modelo de rede neural Mobilenet (HOWARD, A; WANG, W, 2017) é feito, permitindo que, mesmo com poucas imagens, a rede neural possa convergir na classificação dos sinais de LIBRAS.

\section{Organização do trabalho}

Este Trabalho está organizado nos seguintes capítulos:

\begin{itemize}
    \item 2 - Fundamentação Teórica
    \par O capítulo expõe os conceitos necessários para a compreensão do presente trabalho.
    \item 3 - Desenvolvimento
    \par Este capítulo aborda detalhadamente a especificação da biblioteca JavaScript, assim como a metodologia empregada para o desenvolvimento de cada um dos recursos assistivos.
    \item 4 - Resultados
    \par Neste capítulo são expostos, através de aplicações de exemplo, os resultados alcançados com a metodologia aplicada na implementação dos recursos assistivos e na biblioteca.
    \item 5 - Considerações Finais
    \par Este capítulo apresenta as conclusões obtidas com os resultados, assim como uma breve sugestão de trabalhos futuros.
\end{itemize}

% 	\item Capítulo 2: Revisão bibliográfica
% 	\item Capítulo 3: Desenvolvimento % mostra a implementação das tecnologias assistivas com Deep Learning;
% 	\item Capítulo 4: Resultados % expõe os resultados;
% 	\item Capítulo 5: Considerações finais % apresenta as considerações finais
% \end{itemize}
