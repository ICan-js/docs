% ----------------------------------------------------------
% Introdução (exemplo de capítulo sem numeração, mas presente no Sumário)
% ----------------------------------------------------------
\chapter[Introdução]{Introdução}
%\addcontentsline{toc}{chapter}{Introdução}
% ----------------------------------------------------------

% \par O trabalho é motivado pela necessidade / oportunidade de...
% \par Apresentar brevemente o problema para explicar a motiva\~c\~ao para o realizar.
% \par Recomendável a utilização de figuras e/ou tabelas.

Este capítulo demonstra a motivação para o desenvolvimento deste trabalho, os objetivos deste e a metodologia adotada.

\section{Motivação} % Problema
% Dica do Mineda: Colocar mais referências, gastar mesmo aqui.... Isto ajuda a fazer com que a importância do trabalho seja esclarecida.

% Esta motivação está dando a entender que o problema a ser resolvido é outro, NÂO, lembre-se que, objetivo é demonstrar que recursos assistivos criados com Deep Learning podem ser aplicados em tecnologias da WEB!!!!
\par De acordo com o censo do IBGE, realizado em 2010, no Brasil, há cerca de 45 milhões de pessoas com algum tipo de deficiência. E todas estas pessoas necessitam de uma vida independente e de inclusão (SARTORETTO; BERSCH, 2017).

\par Uma das maneiras de permitir que deficientes sejam inclusos na sociedade e tenham uma vida autônoma é com a utilização de  recursos de tecnologias assistivas, estas aliadas a serviços de tecnologia assitiva. Isto porque, estes recursos assitivos deixam de lado a deficiência, e focam nas habilidades presentes no individuo.

\par No Brasil, há uma grande dificuldade de acesso aos recursos de tecnologia assitiva, causadas por diversos fatores, a citar, o alto custo e a necessidade de importação (ANDRIOLI, 2017). O alto custo, na maioria dos casos pode ser justificado pela necessidade de desenvolvimento e construção de equipamentos específicos, o que acaba gerando um alto valor de compra, com equipamentos chegando em valores próximos a 15 mil reais.

\par Por outro lado, tem-se técnicas de \textit{Deep Learning}, que são atualmente o estado-da-arte da solução de problemas com aprendizado de máquina (PONTI, 2017), isto por conta da grande capacidade de generalização diante de diferentes conjuntos de dados. Um de seus grandes benefícios é a possibilidade de alta personalização frente a diferentes tipos de usuários e aplicações.

\par Desta forma, este trabalho foi motivado pela possibilidade da realização de um estudo de casos, onde técnicas de \textit{Deep Learning} são aplicadas para possibilitar a criação de recursos de tecnologias assistivas de baixo custo.

% Section Objetivo Geral!
\section{Objetivo Geral}

\par O objetivo geral deste trabalho é desenvolver uma biblioteca JavaScript que possa levar recursos assistivos desenvolvidos através de técnicas de \textit{Deep Learning} para páginas da \textit{web}.

% \par Implementar recursos de tecnologias assistivas de baixo custo, para usuários com deficiências auditiva e motora, utilizando técnicas de \textit{Deep Learning}

% Section Objetivo Especifico!
\section{Objetivo Espec\'ifico}

\par Para a consecução deste objetivo foram estabelecidos os seguintes objetivos específicos:

% Nestes objetivos está explícito que o objetivo é demonstrar o desenvolvimento de recursos assistivos para web ? ? ? ? ? Acho que não.... 
\begin{itemize}
    \item Desenvolvimento de uma ferramenta que permite a movimentação do \textit{cursor} do \textit{mouse} através de movimentos da cabeça;
    \item Desenvolvimento de uma ferramenta que permite a escrita de textos utilizando LIBRAS;
    \item Integração das ferramentas desenvolvidas em uma biblioteca JavaScript.
\end{itemize}

% Reescrever a metodologia (17/03/2019)
\section{Metodologia} % COMO !! 

\textbf{ESTA SUBSEÇÃO AINDA SERÁ ESCRITA}

% A realização dos objetivos específicos do trabalho é feita através da aplicação de modelos de DL no desenvolvimento das ferramentas. A linguagem de programação empregrada para o desenvolvimento das ferramentas é o Javascript, junto ao \textit{framework} de desenvolvimento de DL \textit{Tensorflow.js}

% Por contar com diferentes estudos de caso, todas as ferramentas são desenvolvidas de maneira modular, a permitir que, no momento da integração entre as ferramentas desenvolvidas, injeções de dependências sejam realizadas para tal.

% Cada uma das ferramentas desenvolvidas, faz a utilização de um modelo de DL, no caso do controle do \textit{mouse} o modelo Posenet (KENDALL, 2015) é utilizado, para facilitar a identificação de pontos faciais do usuário, e permitir que cada gesto seja mapeado em movimentos do \textit{mouse}, o reconhecimento de voz, por sua vez, é feito com o \textit{Web Speech API}, uma \textit{API} livre que facilita a sintetização de som em texto. Por fim, para a tradução de LIBRAS em texto, será utilizado uma rede neural convolucional (LECUN et al, 1998), que apresenta bons resultados na classificação de imagens, junto a um conjunto de imagens de LIBRAS criado pelo autor.

% Retirado apenas para ganhar tempo! <- Esta ainda é a oficial e deve ser levada em consideração ! ! !
% o retreino do modelo de rede neural Mobilenet (HOWARD, A; WANG, W, 2017) é feito, permitindo que, mesmo com poucas imagens, a rede neural possa convergir na classificação dos sinais de LIBRAS.

\section{Organização do trabalho}

Este Trabalho está organizado nos seguintes capítulos:

\begin{itemize}
	\item Capítulo 2: Revisão bibliográfica % apresenta a revisão bibliográfica sobre os temas Tecnologias Assistivas e Deep Learning;
	\item Capítulo 3: Desenvolvimento % mostra a implementação das tecnologias assistivas com Deep Learning;
	\item Capítulo 4: Resultados % expõe os resultados;
	\item Capítulo 5: Considerações finais % apresenta as considerações finais
\end{itemize}
