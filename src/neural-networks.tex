\begin{figure}[H]
    \centering
   \begin{tikzpicture}[
init/.style={
  draw,
  circle,
  inner sep=2pt,
  font=\Huge,
  join = by -latex
},
init2/.style={
  draw,
  circle,
  inner sep=2pt
},
neuron missing/.style={
    draw=none, 
    scale=4,
    text height=0.333cm,
    execute at begin node=\color{black}$\vdots$
  },
squa/.style={
  draw,
  inner sep=2pt,
  font=\Large,
  join = by -latex
},
squa2/.style={
  draw,
  inner sep=2pt,
  font=\Large
},
start chain=2,node distance=13mm
]
\node[on chain=2] 
  (x2) {$x_2$};
%\node[below of=x2] (dots) {$\vdots$} -- (dots) node[right of=dots] (ldots) {$\vdots$};
%\node[below of=2] (dots) {$\vdots$} -- (dots) node[left of=dots] (ldots) {$\vdots$};
\node[on chain=2,init2,join=by o-latex] 
  {$w_{k2}$};
\node[on chain=2,init] (sigma)
  {$\displaystyle\Sigma$};
\node[on chain=2,squa2,label=above:{\parbox{2cm}{\centering Função de \\ ativação}}](te) {$f$};
\node[on chain=2,label=above:Saída] (sa)
  {$y_k$};
\begin{scope}[start chain=1]
\node[on chain=1] at (0,1.5cm) 
  (x1) {$x_1$};
\node[on chain=1,init2,join=by o-latex] 
  (w1) {$w_{k1}$};
\end{scope}
\begin{scope}[start chain=3]
\node[on chain=3] at (0,-1.5cm) 
  (x3) {$x_n$};
\node[on chain=3, init2,label=below:{\parbox{2cm}{\centering Pesos \\ sinápticos}},join=by o-latex] 
  (w3) {$w_{kn}$};
\end{scope}
\node[label=above:\parbox{2cm}{\centering Bias \\ $b$}] at (sigma|-w1) (b) {};

\draw[-latex] (w1) -- (sigma);
\draw[-latex] (w3) -- (sigma);
\draw[o-latex] (b) -- (sigma);
\draw[-latex] (sigma) -- (te) node[midway,sloped, above]{$v_k$};
\draw[-latex] (te) -- (sa) node[midway,sloped, above]{};
\draw[decorate,decoration={brace,mirror}] (x1.north west) -- node[left=10pt] {Entradas} (x3.south west);
\end{tikzpicture}
    \caption{Neurônio Artificial}
    % \fonte{Adaptado de \citeonline{haykin2009neural}}
    \label{fig:modelo_neuronio}
\end{figure}